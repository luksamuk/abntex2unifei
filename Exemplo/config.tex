% Definições do documento
\documentclass[
	% -- opções da classe memoir --
	12pt,				% tamanho da fonte
	%openright,			% capítulos começam em pág ímpar (insere página vazia caso preciso)
	%twoside,			% para impressão em verso e anverso. Oposto a oneside
	oneside,	
	a4paper,			% tamanho do papel
	% -- opções da classe abntex2 --
	chapter=TITLE,		% títulos de capítulos convertidos em letras maiúsculas
	hyphens,            % Hífens nos links
	% -- opções do pacote babel --
	english,			% idioma adicional para hifenização
	french,				% idioma adicional para hifenização
	spanish,			% idioma adicional para hifenização
	brazil,				% o último idioma é o principal do documento
	]{abntex2}




% Pacotes
% Principais
\usepackage[T1]{fontenc}		% Selecao de codigos de fonte.
\usepackage[utf8]{inputenc}		% Codificacao do documento (conversão automática dos acentos)
\usepackage{indentfirst}		% Indenta o primeiro parágrafo de cada seção.
\usepackage{color}				% Controle das cores
\usepackage{graphicx}			% Inclusão de gráficos
\usepackage{microtype} 			% para melhorias de justificação
\usepackage{titlesec}           % para definir o formato do título
\usepackage{enumitem}           % Para realizar enumerações com ítens

% --------------
\usepackage{../abntex2unifei}   % Adequações à Unifei. Usando localização relativa.
% --------------

% Citações
\usepackage[brazilian,hyperpageref]{backref}	 % Paginas com as citações na bibl
\usepackage[alf,abnt-emphasize=bf]{abntex2cite}  % Citações padrão ABNT, com referências em negrito

% Outros
\usepackage{gensymb}                             % Símbolo de graus

\usepackage{lipsum}				% para geração de dummy text




% Configurações de pacotes

% Configurações do pacote backref
% Usado sem a opção hyperpageref de backref
\renewcommand{\backrefpagesname}{Citado na(s) página(s):~}
% Texto padrão antes do número das páginas
\renewcommand{\backref}{}
% Define os textos da citação
\renewcommand*{\backrefalt}[4]{
	\ifcase #1 %
		Nenhuma citação no texto.%
	\or
		Citado na página #2.%
	\else
		Citado #1 vezes nas páginas #2.%
	\fi}%





% Configurações de aparência do PDF final

% alterando o aspecto da cor azul
\definecolor{blue}{RGB}{41,5,195}

% Espaçamentos entre parágrafo
% O tamanho do parágrafo é dado por:
\setlength{\parindent}{1.3cm}
% Controle do espaçamento entre um parágrafo e outro:
\setlength{\parskip}{0.3cm}  % tente também \onelineskip
