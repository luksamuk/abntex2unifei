% Configuração geral
% Definições do documento
\documentclass[
	% -- opções da classe memoir --
	12pt,				% tamanho da fonte
	%openright,			% capítulos começam em pág ímpar (insere página vazia caso preciso)
	%twoside,			% para impressão em verso e anverso. Oposto a oneside
	oneside,	
	a4paper,			% tamanho do papel
	% -- opções da classe abntex2 --
	chapter=TITLE,		% títulos de capítulos convertidos em letras maiúsculas
	hyphens,            % Hífens nos links
	% -- opções do pacote babel --
	english,			% idioma adicional para hifenização
	french,				% idioma adicional para hifenização
	spanish,			% idioma adicional para hifenização
	brazil,				% o último idioma é o principal do documento
	]{abntex2}




% Pacotes
% Principais
\usepackage[T1]{fontenc}		% Selecao de codigos de fonte.
\usepackage[utf8]{inputenc}		% Codificacao do documento (conversão automática dos acentos)
\usepackage{indentfirst}		% Indenta o primeiro parágrafo de cada seção.
\usepackage{color}				% Controle das cores
\usepackage{graphicx}			% Inclusão de gráficos
\usepackage{microtype} 			% para melhorias de justificação
\usepackage{titlesec}           % para definir o formato do título
\usepackage{enumitem}           % Para realizar enumerações com ítens

% --------------
\usepackage{../abntex2unifei}   % Adequações à Unifei. Usando localização relativa.
% --------------

% Citações
\usepackage[brazilian,hyperpageref]{backref}	 % Paginas com as citações na bibl
\usepackage[alf]{abntex2cite}	                 % Citações padrão ABNT

% Outros
\usepackage{gensymb}                             % Símbolo de graus

\usepackage{lipsum}				% para geração de dummy text




% Configurações de pacotes

% Configurações do pacote backref
% Usado sem a opção hyperpageref de backref
\renewcommand{\backrefpagesname}{Citado na(s) página(s):~}
% Texto padrão antes do número das páginas
\renewcommand{\backref}{}
% Define os textos da citação
\renewcommand*{\backrefalt}[4]{
	\ifcase #1 %
		Nenhuma citação no texto.%
	\or
		Citado na página #2.%
	\else
		Citado #1 vezes nas páginas #2.%
	\fi}%





% Configurações de aparência do PDF final

% alterando o aspecto da cor azul
\definecolor{blue}{RGB}{41,5,195}

% Espaçamentos entre parágrafo
% O tamanho do parágrafo é dado por:
\setlength{\parindent}{1.3cm}
% Controle do espaçamento entre um parágrafo e outro:
\setlength{\parskip}{0.3cm}  % tente também \onelineskip


% Informações de dados para CAPA e FOLHA DE ROSTO
\titulo{Título do Trabalho}
\autor{Zé da Silva}
\local{Itabira}
\data{2016}
\instituicao{Universidade Federal de Itajubá - \textit{Campus} Itabira}
\tipotrabalho{Trabalho de Conclusão de Curso}
\orientador{Fessor dos Santos}
% O preambulo deve conter o tipo do trabalho, o objetivo, 
% o nome da instituição e a área de concentração 
\preambulo{Trabalho de Conclusão de Curso de um discente da Universidade Federal de Itajubá - \textit{Campus} Itabira}

% informações do PDF
\makeatletter
\hypersetup{
     	%pagebackref=true,
		pdftitle={\@title}, 
		pdfauthor={\@author},
    	pdfsubject={\imprimirpreambulo},
	    pdfcreator={Zé da Silva},
		pdfkeywords={abntex2}{abntex2unifei}{latex}{texto}, 
		colorlinks=true,       		% false: boxed links; true: colored links
    	linkcolor=black,          	% color of internal links
    	citecolor=black,        	% color of links to bibliography
    	filecolor=magenta,      	% color of file links
		urlcolor=blue,
		bookmarksdepth=4
}
\makeatother


% Compila o indice
\makeindex


\begin{document}

	% Elementos pré-textuais
	\pretextual
	% Os elementos a seguir aparecerão antes do início do artigo em si.

% Espaçamento de 1,5cm por linha
\OnehalfSpacing

% Capa
\imprimircapa

% Folha de rosto
\imprimirfolhaderosto

% Folha de aprovação
% COMANDO EXCLUSIVO DA ABNTEX2UNIFEI.
\imprimirfolhadeaprovacao{Este Trabalho de Pesquisa foi julgado, como requisito parcial, para aprovação na disciplina Xxxxxxx da Engenharia Xxxxxxxxxx da Universidade Federal de Itajubá – \textit{campus} Itabira.}


% Espaço reservado a dedicatórias
\begin{dedicatoria}
\null
\vfill
Espaço reservado à dedicatória, elemento opcional destinado a homenagear pessoas importantes na vida do autor do trabalho.
\end{dedicatoria}


% Agradecimentos
\begin{agradecimentos}[Agradecimentos]
Elemento opcional que é utilizado para agradecer a pessoas ou instituições que contribuíram com a realização do trabalho.
\end{agradecimentos}

% Epígrafe
\begin{epigrafe}
\null
\vfill
Espaço reservado à epígrafe, elemento opcional, elaborado conforme a ABNT NBR 10520, em que se transcreve uma citação literal, com autoria, referente ao assunto abordado no trabalho.
\end{epigrafe}


% Resumo
\begin{resumo}
Elemento obrigatório que apresenta uma sequência de frases objetivas e concisas, de 150 a 500 palavras, seguido de palavras-chave, as quais representam o conteúdo do trabalho. Geralmente apresenta: tema central, objetivo da pesquisa, aporte teórico, metodologia empregada, resultados e conclusões (tudo bem sucinto e sem citações).


Palavras-chave: 3 a 5 palavras-chave. Separadas entre si por ponto.
\end{resumo}

\begin{resumo}[Abstract]
Apresentação, em inglês, do resumo que apresente o conteúdo de todo o trabalho (tema central, objetivo da pesquisa, aporte teórico, metodologia empregada, resultados e conclusões).


Keywords: 3 a 5 palavras-chave. Separadas entre si por ponto. Todas em inglês.
\end{resumo}



% Lista de Figuras
\pdfbookmark[0]{\listfigurename}{lof}
\listoffigures*
\cleardoublepage

% Lista de Tabelas
\pdfbookmark[0]{\listtablename}{lot}
\listoftables*
\cleardoublepage

% Lista de Abreviaturas e Siglas
\begin{siglas}
  \item[Embrapa] Empresa Brasileira de Pesquisa Agropecuária
  \item[IBGE] Instituto Brasileiro de Geografia e Estatística
\end{siglas}

% Lista de Símbolos
\begin{simbolos}
  \item[$ \Gamma $] Letra grega Gama
  \item[$ \Lambda $] Lambda
  \item[$ \zeta $] Letra grega minúscula zeta
  \item[$ \in $] Pertence
  \item[O(n)] Ordem de um algoritmo
  \item[©] Copyright
\end{simbolos}

% Sumário
\pdfbookmark[0]{\contentsname}{toc}
\tableofcontents*
\cleardoublepage


	% Elementos textuais
	\textual
	
	% Inclui fontes não-citadas nas referências,
	% apenas para que apareçam como exemplo.
	\nocite{cepel}
	\nocite{aneel}
	\nocite{folha}
    \nocite{sicp}
	
	% Introdução	
	\chapter{Introdução}
% Nada de mais por aqui.
\lipsum[1-3]

	% Referencial Teórico
	\phantompart
	\chapter{Referencial Teórico}

\section{Primeira Seção}

% Um exemplo de citação em linha:
De acordo com \citeonline{haykin}, \lipsum[1-4]

	% Exemplo de figura.
	\begin{figure}[!ht]
		\centering
		\caption{Nome desta figura.}
		% Você pode inserir sua figura no lugar deste
		% fbox.
		\fbox{\rule[5cm]{\textwidth}{0cm}}
		\begin{minipage}{\textwidth}
			\centering
			{\small Fonte: Fonte da figura.}\par
		\end{minipage}
		\label{fig:irradiacao}
	\end{figure}
	
\section{Segunda Seção}
\lipsum[5-8]

\section{Terceira Seção}
\lipsum[9-10]
	
\section{Quarta Seção}
\lipsum[11]
	
	% Exemplo de figura.
	\begin{figure}[!ht]
		\centering
		\caption{Nome desta figura.}
		% Você pode inserir sua figura no lugar deste
		% fbox.
		\fbox{\rule[6cm]{\textwidth}{0cm}}
		\begin{minipage}{\textwidth}
			\centering
			{\small Fonte: Fonte desta figura.}\par
		\end{minipage}
		\label{fig:aquecedor1}
	\end{figure}
	
\lipsum[12]
	
	% Exemplo de figura.
	\begin{figure}[!ht]
		\centering
		\caption{Nome desta figura.}
		% Você pode inserir sua figura no lugar deste
		% fbox.
		\fbox{\rule[5cm]{\textwidth}{0cm}}
		\begin{minipage}{\textwidth}
			\centering
			{\small Fonte: Fonte desta figura.}\par
		\end{minipage}
		\label{fig:aquecedor2}
	\end{figure}
	
\lipsum[13]
	
\section{Quinta seção}
\lipsum[14-17]


	% Metodologia
	\phantompart
	\chapter{Metodologia}

\lipsum[1]
	
\section{Primeira Seção}

\lipsum[2]

	% Exemplo de tabela, para referências futuras.
	\begin{table}[!ht]
	\IBGEtab{
		% Nome da tabela
		\caption{Orçamento}
		% Rótulo da tabela
		\label{tab:orcamento}
		}
		% Conteúdo da tabela
		{
		\begin{tabular}{c|l|r}
		\toprule
		
		Qtd. & Material & Preço\\
		
		\midrule
		
		2  & Barras (superior e inferior) com tubo de 32 x 670 mm & R\$ 29,90\\
		20 & Barras de interligação com tubo de 200 x 1440 mm     & R\$ 85,90\\
		1  & Cola 'Plexus 310'                                    & R\$ 59,00\\
		1  & Fita crepe                                           & R\$ 3,59\\
		1  & Tinta esmalte sintético preto fosco (madeira ou metais), ou tinta preto fosca automotiva & R\$ 35,90\\
		2  & Cap soldáveis de 32 mm                               & R\$ 4,60\\
		1  & Joelho soldável de 90$^{\circ}$ de 32 mm             & R\$ 3,10\\
		1  & Te (T) soldável de 32 mm                             & R\$ 4,00\\
		1  & Adaptador soldável com bolsa e rosca para registro de 32 mm x 1'' (polegada) & R\$ 1,58\\
		1  & Cap roscável 1'' (polegada)                          & Não encontrado\\
		1  & Coletor                                              & Não encontrado\\
		1  & Reservatório (250 L)                                 & R\$ 158,58\\
		1  & Furadeira com serra tipo copo                        & R\$ 98,90\\
		1  & Boia de nível reservatório                           & R\$ 27,90\\
		1  & Boia pescador                                        & R\$ 26,90\\
		
		\midrule
		
		   & Preço total SEM reservatório                         & R\$ 381,27\\
		   & Preço total COM reservatório                         & R\$ 535,85\\
		   
		\bottomrule
		\end{tabular}
		}
		% Conteúdo da legenda da tabela
		{
			%\fonte{}
			\nota{Nota adicional de rodapé.}
		}
	\end{table}

\section{Segunda Seção}
\lipsum[3]

\section{Terceira Seção}
	\begin{figure}[!ht]
		\centering
		\fbox{\rule[10cm]{12cm}{0cm}}
		\caption{Legenda desta figura.}
		\begin{minipage}{\textwidth}
			\centering
			{\small Fonte: Fonte desta figura.}\par
		\end{minipage}
		\label{fig:kitexposicao}
	\end{figure}

	
	% Conclusão
	\phantompart
	\chapter{Considerações Finais}
% Nada de mais por aqui.
\lipsum[1-2]


	% Elementos pós-textuais
	\postextual
	% Os elementos a seguir aparecerão ao final do artigo.

% Referências bibliográficas
\bibliography{refs}

% Glossário
% Consulte o manual da classe abntex2 para orientações sobre o glossário.
%\glossary

% Apêndices
\begin{apendicesenv}
% Imprime uma página indicando o início dos apêndices
\partapendices
%
\chapter{Quisque libero justo}
\lipsum[50]

\chapter{Nullam elementum urna vel imperdiet}
\lipsum[55-57]
\end{apendicesenv}

% Anexos
\begin{anexosenv}
% Imprime uma página indicando o início dos anexos
\partanexos

\chapter{Lorem Ipsum}
\lipsum[1-4]

\chapter{Nulla malesuada porttitor diam}
\lipsum[3-6] 

\end{anexosenv}


% INDICE REMISSIVO
\phantompart
\printindex


\end{document}
