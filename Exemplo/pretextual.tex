% Os elementos a seguir aparecerão antes do início do artigo em si.

% Espaçamento de 1,5cm por linha
\OnehalfSpacing

% Capa
\imprimircapa

% Folha de rosto
\imprimirfolhaderosto

% Folha de aprovação
% COMANDO EXCLUSIVO DA ABNTEX2UNIFEI.
\imprimirfolhadeaprovacao{Este Trabalho de Pesquisa foi julgado, como requisito parcial, para aprovação na disciplina Xxxxxxx da Engenharia Xxxxxxxxxx da Universidade Federal de Itajubá – \textit{campus} Itabira.}


% Espaço reservado a dedicatórias
\begin{dedicatoria}
\null
\vfill
Espaço reservado à dedicatória, elemento opcional destinado a homenagear pessoas importantes na vida do autor do trabalho.
\end{dedicatoria}


% Agradecimentos
\begin{agradecimentos}[Agradecimentos]
Elemento opcional que é utilizado para agradecer a pessoas ou instituições que contribuíram com a realização do trabalho.
\end{agradecimentos}

% Epígrafe
\begin{epigrafe}
\null
\vfill
Espaço reservado à epígrafe, elemento opcional, elaborado conforme a ABNT NBR 10520, em que se transcreve uma citação literal, com autoria, referente ao assunto abordado no trabalho.
\end{epigrafe}


% Resumo
\begin{resumo}
Elemento obrigatório que apresenta uma sequência de frases objetivas e concisas, de 150 a 500 palavras, seguido de palavras-chave, as quais representam o conteúdo do trabalho. Geralmente apresenta: tema central, objetivo da pesquisa, aporte teórico, metodologia empregada, resultados e conclusões (tudo bem sucinto e sem citações).


Palavras-chave: 3 a 5 palavras-chave. Separadas entre si por ponto.
\end{resumo}

\begin{resumo}[Abstract]
Apresentação, em inglês, do resumo que apresente o conteúdo de todo o trabalho (tema central, objetivo da pesquisa, aporte teórico, metodologia empregada, resultados e conclusões).


Keywords: 3 a 5 palavras-chave. Separadas entre si por ponto. Todas em inglês.
\end{resumo}



% Lista de Figuras
\pdfbookmark[0]{\listfigurename}{lof}
\listoffigures*
\cleardoublepage

% Lista de Tabelas
\pdfbookmark[0]{\listtablename}{lot}
\listoftables*
\cleardoublepage

% Lista de Abreviaturas e Siglas
\begin{siglas}
  \item[Embrapa] Empresa Brasileira de Pesquisa Agropecuária
  \item[IBGE] Instituto Brasileiro de Geografia e Estatística
\end{siglas}

% Lista de Símbolos
\begin{simbolos}
  \item[$ \Gamma $] Letra grega Gama
  \item[$ \Lambda $] Lambda
  \item[$ \zeta $] Letra grega minúscula zeta
  \item[$ \in $] Pertence
  \item[O(n)] Ordem de um algoritmo
  \item[©] Copyright
\end{simbolos}

% Sumário
\pdfbookmark[0]{\contentsname}{toc}
\tableofcontents*
\cleardoublepage