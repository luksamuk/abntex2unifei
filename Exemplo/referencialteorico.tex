\chapter{Referencial Teórico}

\section{Primeira Seção}

% Um exemplo de citação em linha:
De acordo com \citeonline{haykin}, \lipsum[1-4]

	% Exemplo de figura.
	\begin{figure}[!ht]
		\centering
		\caption{Nome desta figura.}
		% Você pode inserir sua figura no lugar deste
		% fbox.
		\fbox{\rule[5cm]{\textwidth}{0cm}}
		\begin{minipage}{\textwidth}
			\centering
			{\small Fonte: Fonte da figura.}\par
		\end{minipage}
		\label{fig:irradiacao}
	\end{figure}
	
\section{Segunda Seção}
\lipsum[5-8]

\section{Terceira Seção}
\lipsum[9-10]
	
\section{Quarta Seção}
\lipsum[11]
	
	% Exemplo de figura.
	\begin{figure}[!ht]
		\centering
		\caption{Nome desta figura.}
		% Você pode inserir sua figura no lugar deste
		% fbox.
		\fbox{\rule[6cm]{\textwidth}{0cm}}
		\begin{minipage}{\textwidth}
			\centering
			{\small Fonte: Fonte desta figura.}\par
		\end{minipage}
		\label{fig:aquecedor1}
	\end{figure}
	
\lipsum[12]
	
	% Exemplo de figura.
	\begin{figure}[!ht]
		\centering
		\caption{Nome desta figura.}
		% Você pode inserir sua figura no lugar deste
		% fbox.
		\fbox{\rule[5cm]{\textwidth}{0cm}}
		\begin{minipage}{\textwidth}
			\centering
			{\small Fonte: Fonte desta figura.}\par
		\end{minipage}
		\label{fig:aquecedor2}
	\end{figure}
	
\lipsum[13]
	
\section{Quinta seção}
\lipsum[14-17]
